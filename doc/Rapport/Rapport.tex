\documentclass[11pt]{report}
\usepackage[margin=2.5cm]{geometry}
\usepackage[french]{babel}
\usepackage[T1]{fontenc}
\usepackage[explicit]{titlesec}
\usepackage{times}
\usepackage{fancyhdr}
\usepackage{graphicx}
\usepackage{ucs}
\usepackage[utf8x]{inputenc}
\usepackage{awesomebox}
\usepackage{fontawesome5}
\setmainfont{Liberation Serif}

\titleformat{\chapter}[display]{\Huge}{\thechapter. #1}{20pt}{\small}

\titlespacing{\chapter}{0pt}{.1cm}{.1cm}

\lhead[\rightmark]{\rightmark}
\chead[]{}
\rhead[\thepage]{\thepage}

\lfoot[]{}
\cfoot[\thepage]{\thepage}
\rfoot[]{}

\renewcommand{\headrulewidth}{0.5pt}

\pagestyle{fancy}

\begin{document}
\begin{titlepage}
   \begin{center}
       \vspace*{5cm}

       \Huge\textbf{Rapport}

       \vspace{0.5cm}
       \Large


       Moteur 3D - 7Physics


       \includegraphics[width=2cm]{./logo.png}

       \vspace{1cm}

       \large
       \textbf{Équipe 3 : Noa AMMIRATI, Fanny DELNONDEDIEU, Quentin GENDARME, Pierre LOTTE, Théo PIROUELLE, Éléa TURC}

       \vfill

       \includegraphics[width=15cm]{./enseeiht.jpeg}

       \vspace{2cm}

       ENSEEIHT\\
       Département Sciences du Numérique\\
       1APP SN 2020-2021


   \end{center}
\end{titlepage}


\tableofcontents


\chapter{Introduction}

L'idée de ce projet est de réaliser un moteur 3D permettant de réaliser des simulations de notions de 
physique élémentaires telles que la gravité, les collisions etc.

\section{....}

\notebox{...}




\chapter{Principales fonctionnalités}

\section{Sprint 0}

\subsection{Afficher une scène 3D}
Une des premières fonctionnalités à implanter a été la création d'une scène 3D. 
Cette scène 3D est constituée d'un sol et d'un ciel afin de permettre à l'utilisateur d'avoir les notions
d'espace et de profondeur.


\section{Sprint 1}
\subsection{Manipuler des objets 3D}
\subsubsection{Ajouter un objet 3D}
\subsubsection{Supprimer un objet 3D}

\chapter{Découpage de l'application}


\chapter{Diagramme de classe}


\chapter{Principaux choix}

\subsection{Mise en place du projet}
L'objectif du sprint 0 à été de mettre en place le projet. Pour cela, il a tout d'abord fallu 
déterminer les différents objectifs et les différents besoins utilisateur. Ensuite, l'équipe a défini
les outils à utiliser concernant la gestion de projet. Pour finir, le projet a été structuré en différents répertoires 
et l'environnement de travail a été configuré pour chaque membre de l'équipe.

\subsection{Création de la maquette IHM}
A l'aide de l'outil Figma, une maquette IHM a été réalisée afin de concrétiser les idées des membres de l'équipe
et de représenter concrètement le concept à construire.

\subsection{Création d'un diagramme de classe}

\section{Conception}

\section{Réalisation}

\section{Problèmes rencontrés et solutions apportées}


\chapter{Organisation de l'équipe et mise en oeuvre des méthodes agiles}

\end{document}
