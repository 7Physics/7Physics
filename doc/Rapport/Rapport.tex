\documentclass[11pt]{report}
\usepackage[margin=2.5cm]{geometry}
\usepackage[french]{babel}
\usepackage[T1]{fontenc}
\usepackage[explicit]{titlesec}
\usepackage{times}
\usepackage{fancyhdr}
\usepackage{graphicx}
\usepackage{ucs}
\usepackage[utf8x]{inputenc}
\usepackage{awesomebox}
\usepackage{fontawesome5}
\setmainfont{Liberation Serif}

\titleformat{\chapter}[display]{\Huge}{\thechapter. #1}{20pt}{\small}

\titlespacing{\chapter}{0pt}{.1cm}{.1cm}

\lhead[\rightmark]{\rightmark}
\chead[]{}
\rhead[\thepage]{\thepage}

\lfoot[]{}
\cfoot[\thepage]{\thepage}
\rfoot[]{}

\renewcommand{\headrulewidth}{0.5pt}

\pagestyle{fancy}

\begin{document}
\begin{titlepage}
   \begin{center}
       \vspace*{5cm}

       \Huge\textbf{Rapport}

       \vspace{0.5cm}
       \Large


       Moteur 3D - 7Physics


       \includegraphics[width=2cm]{./logo.png}

       \vspace{1cm}

       \large
       \textbf{Équipe 3 : Noa AMMIRATI, Fanny DELNONDEDIEU, Quentin GENDARME, Pierre LOTTE, Théo PIROUELLE, Éléa TURC}

       \vfill

       \includegraphics[width=15cm]{./enseeiht.jpeg}

       \vspace{2cm}

       ENSEEIHT\\
       Département Sciences du Numérique\\
       1APP SN 2020-2021


   \end{center}
\end{titlepage}


\tableofcontents


\chapter{Introduction}

L'idée de ce projet est de réaliser un moteur 3D permettant de réaliser des simulations de notions de 
physique élémentaires telles que la gravité, les collisions etc.

\chapter{Principales fonctionnalités}

\section{Sprint 0}

\subsection{Afficher une scène 3D}
Une des premières fonctionnalités à implanter a été la création d'une scène 3D. 
Cette scène 3D est constituée d'un sol représentant une grille blanche sur fond gris et d'un ciel bleu.
Cette scène a pour but de permettre à l'utilisateur de mieux comprendre l'orientation des objets,le placement
de sa caméra, etc. Cela permettra alors, lors de simulation physiques, de mieux comprendre les résultats de celles-ci.

\section{Sprint 1}
\subsection{Manipuler des objets 3D}
\subsubsection{Ajouter un objet 3D}

\subsubsection{Supprimer un objet 3D}

\chapter{Découpage de l'application}

Afin de séparer les rôles le plus possibles et permettre l'utilisation, à termes, de parties de l'application
indépendamment, nous avons commencer par créer 4 répertoires git différent. Cela permet alors de n'utiliser que
les fonctions de rendus dans une librairie minimaliste si besoin et de même pour la partie simulation physique.\newline

Dans chaque répertoire était alors développée une des 4 briques de l'application finale:\newline

\begin{itemize}
  \item \textbf{7Physics}: Répertoire contenant le code de l'interface graphique. Ce répertoire se base sur les 3 autres
        et contient l'exécutable final. Son seul rôle est d'articuler les différentes fonctionnalités du moteur et ce
        dans une interface graphique agréable à utiliser.\newline
  \item\textbf{Common}: Répertoire contenant tout le code commun est nécessaire au fonctionnement des autres. Dans ce
        répertoire, nous pouvons trouver la représentation des formes, la définition d'une classe représentant la position,
        un logger permettant de tracer les appels systèmes, etc.\newline
  \item \textbf{Renderer}: Répertoire contenant uniquement les interactions avec le contexte OpenGL. Afin de fonctionner,
        ce répertoire ne nécessite que l'utilisation du répertoire Common. Il est alors tout à fait pensable de n'utiliser
        que cette partie de l'application et de lui demander de fournir un affichage en 3 dimensions à partir de classes
        créées par l'utilisateur.\newline
  \item \textbf{Engine}: Répertoire permettant la simulation physique. Ce répertoire ne nécessite que l'utilsiation
        de Common pour fonctionner. En effet, tout comme Renderer, il est tout à fait pensable de n'utiliser que ce répertoire
        comme une librairie de calcul physique simple.\newline
\end{itemize}

\chapter{Diagramme de classe}


\chapter{Principaux choix}

\section{Création de la maquette IHM}
A l'aide de l'outil Figma, une maquette IHM a été réalisée afin de concrétiser les idées des membres de l'équipe
et de représenter concrètement le concept à construire.

\section{Création d'un diagramme de classe}

\section{Conception}

\section{Réalisation}

\section{Problèmes rencontrés et solutions apportées}

\chapter{Organisation de l'équipe et mise en oeuvre des méthodes agiles}

\section{Mise en place du projet}
L'objectif du sprint 0 à été de mettre en place le projet. Pour cela, il a tout d'abord fallu
déterminer les différents objectifs et les différents besoins utilisateur au travers de User stories. Ensuite, l'équipe a défini
les outils à utiliser concernant la gestion de projet, la gestion du code et la communication au sein de l'équipe.
Pour finir, le projet a été structuré en différents répertoires au sein de l'organsation GitHub créée pour le projet long
et l'environnement de travail a été configuré pour chaque membre de l'équipe.\newline

Une fois cela fait, nous avons pu commencer à réaliser les premières User Stories.

\section{Utilisation des cérémonies agiles}

Lors du déroulement du projet, nous avons mis en place 4 types de cérémonies des méthodes agiles.

\subsection{Sprint planning}

Au début de chaque sprint nous avons commencé par se réunir tous ensemble pour faire un point sur l'état d'esprit de tout le monde.
Nous échangions alors sur le ressenti par rapport au projet et l'état d'avancement de celui-ci, les difficultés eprouvées, etc. Cette première étape
nous permettait alors de s'assurer de la bonne entente entre les membres de l'équipe et du fait que personne ne se sente seul ou isolé du groupe.

Une fois cela fait, nous continuions par choisir les prochaines User Stories à implémenter au sein de l'application en se basant sur la vélocité
de l'équipe lors du sprint précédent.\newline

Le sprint était alors prêt à commencer.

\subsection{Daily scrum}

Lors du déroulement de chaque Sprint, nous avions des réunions régulières. Qui malgrés ce que le nom
laisse penser, n'avait lieu que 2 à 3 fois par semaine. En effet, ayant d'autres matières demandant du travail
nous n'avions pas toujours le temps d'avancer chaque jour. Nous avons donc décider de nous réunir tous les 2 à 3 jours
pour faire le point sur l'avancement de chacun depuis le dernier daily scrum, les choix de conception et d'implémentation que
nous devions revoir en cours de projet, etc.

\subsection{Mid-sprint review}

Au milieu de chaque sprint, nous nous réunissions afin de discuter de l'état courant du sprint. Le but était alors, dans le cas
où tout avait déjà été fait, d'ajouter de nouvelles User Stories à réaliser, ou, dans le cas contraire, d'alléger la charge
de travail prévue.

Cette cérémonie nous permettait alros d'être sûr que notre charge de travail était réalisable lors du temps imparti
et de s'assurer que tout se déroulait comme prévu

\subsection{Sprint Review}

Enfin, pour clôturer le sprint, nous organisions une réunion ayant pour but de revenir sur les évènements du sprint.
Cette cérémonie était cruciale car elle nous permettait notamment de calculer la vélocité à utiliser pour le prochain sprint,
discuter des difficultés rencontrées et définir les défauts du sprint réalisé que nous devrons corriger lors du prochain.


\end{document}
